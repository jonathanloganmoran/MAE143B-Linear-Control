\documentclass{article}
\usepackage[landscape]{geometry}
\usepackage{url}
\usepackage{multicol}
\usepackage{amsmath}
\usepackage{esint}
\usepackage{amsfonts}
\usepackage{tikz}
\usetikzlibrary{decorations.pathmorphing}
\usepackage{amsmath,amssymb}
\usepackage{mathrsfs}

\usepackage{colortbl}
\usepackage{xcolor}
\usepackage{mathtools}
\usepackage{amsmath,amssymb}
\usepackage{enumitem}
\makeatletter

\newcommand*\bigcdot{\mathpalette\bigcdot@{.5}}
\newcommand*\bigcdot@[2]{\mathbin{\vcenter{\hbox{\scalebox{#2}{$\m@th#1\bullet$}}}}}
\makeatother

\title{MAE 143B Equation Sheet}
\author{Jonathan L. Moran (ax007859)}
\usepackage[brazilian]{babel}
\usepackage[utf8]{inputenc}

\advance\topmargin-.8in
\advance\textheight3in
\advance\textwidth3in
\advance\oddsidemargin-1.5in
\advance\evensidemargin-1.5in
\parindent0pt
\parskip2pt
\newcommand{\hr}{\centerline{\rule{3.5in}{1pt}}}
%\colorbox[HTML]{e4e4e4}{\makebox[\textwidth-2\fboxsep][l]{texto}

\begin{document}
\begin{center}{\huge{\textbf{MAE 143B Equation Sheet}}}\\
\end{center}
\begin{multicols*}{3}

\tikzstyle{mybox} = [draw=black, fill=white, very thick,
    rectangle, rounded corners, inner sep=10pt, inner ysep=10pt]
\tikzstyle{fancytitle} =[fill=black, text=white, font=\bfseries]

%------------ Galileo's Inclined Plane Model ---------------
\begin{tikzpicture}
\node [mybox] (box){%
    \begin{minipage}{0.3\textwidth}
        $h$, fall height (ft)\\
        $d$, horizontal travel distance (in)${}^{*}$, \\
        $l$, length of inclined plane (ft), \\
        $y$, vertical distance${}^{*}$. \\
        \\
    MATLAB commands:
    \\
    \small{
    	\begin{tabular}{lp{4cm} l} \hline
    	\textit{Fitting linear model} &     $f = \texttt{fit(d, h, 'poly1')}$. \\ \hline
    	\textit{Fitting quadratic model} &  $f = \texttt{fit(d, h, 'poly2')}$, \\ &
    	                                    for $\texttt{h} = l sin(\theta)$. \\ \hline
    \end{tabular}}
    \\
    \\
    Estimating $y$:
    \\
    \small{
    	\begin{tabular}{lp{4cm} l} \hline
    	\textit{Conservation of energy} &   $\frac{1}{2}mv_{x}^2 = mgh$, \\ &
    	                                    $\Rightarrow v_x = \sqrt{2gh}$. \\ \hline
    	\textit{Vertical motion} &          $y = \frac{1}{2}gt^2$, \\ &
    	                                    $\Rightarrow t = \sqrt{2y / g}$. \\ \hline
    	\textit{Horizontal motion} &        $d = v_xt$, \\ &
    	                                    $d = \sqrt{2gh} \ctimes \sqrt{2y/g} $, \\ &
    	                                    $\Rightarrow d^2 / 4 = yh$. \\ \hline
    \end{tabular}}

    ${}^{*}$ units converted to feet.
    \end{minipage}
};
%------------ Galileo Header ---------------------
\node[fancytitle, right=10pt] at (box.north west) {P1.10 Galileo's Inclined Plane Problem};
\end{tikzpicture}

%------------ Elevator Model ---------------
\begin{tikzpicture}
\node [mybox] (box){%
    \begin{minipage}{0.3\textwidth}
    \small{
    	\begin{tabular}{lp{5cm} l} \hline
    	\textit{Wheels} &   $J_1\dot{\omega} + b_1{\omega} = \tau + (f_1 - f_2)r$, \\ & 
    	                    $J_2\dot{\omega} + b_2{\omega} = \tau + (f_4 - f_3)r$. \\ \hline
		\textit{Forces from $J_1$} &    $f_1 + m_1\dot{v_1} = m_1g + f_3$. \\ \hline
        \textit{Forces from $J_2$} &    $f_2 + m_2\dot{v_2} = m_2g + f_4$. \\ \hline
        \textit{Angular velocity}  &    $v_1 = r\omega, v_2 = -r\omega$. \\ \hline
	\end{tabular}}

    \begin{equation}
        (J_1 + J_2 + r^2(m_1 + m_2))\dot{\omega} + (b_1 + b_2)\omega =  \tau + gr(m_1 - m_2) \\
    \end{equation}
    \\ ODE solution:
    \begin{equation}
        v_1(t) = \Tilde{v_1}(1 - e^{\lambda t}) + v_1(0)e^{\lambda t}, \\
    \end{equation}
    Furthermore:
    \begin{align*}
        \Bar{v_1} = v_1(t)_{t\rightarrow\infty}, TC = -\frac{1}{\lambda}.
    \end{align*}

    \small{
    	\begin{tabular}{lp{5cm} l} \hline
    	\textit{Steady state error} &   $e_{SS} = \Bar{v_1} - v_{SS} = \Bar{v_1} - \Tilde{v_1}$. \\ \hline
		\textit{Open loop} &    $\lambda_{OL} = -\frac{b}{a}$, \\ & 
                                $\Bar{v_1} = \frac{r}{b}[\tau + gr(m_1 - m_2)]$, \\ &
                                $TC_{OL} = -\frac{1}{\lambda_{OL}}$. \\ \hline
        \textit{Closed loop} &  $\tau(t) = K(\Bar{v_1} - v_1(t))$ \textit{controller} \\ &
                                $\lambda_{CL} = -\frac{(b + Kr)}{a}$, \\ & 
                                $\Tilde{v_1} = \frac{r}{b + Kr}[K\Bar{v_1} + gr(m_1 - m_2)]$, \\ &
                                $TC_{OL} = -\frac{1}{\lambda_{OL}}$, \\ &
                                $K = \frac{a - 5b}{5r}$.
	\end{tabular}}
    
    
    
    \end{minipage}
};
%------------ Elevator Header ---------------------
\node[fancytitle, right=10pt] at (box.north west) {P2.10 Elevator Problem};
\end{tikzpicture}


%------------ Laplace Transforms Content ---------------
\begin{tikzpicture}
\node [mybox] (box){%
    \begin{minipage}{0.3\textwidth}
    $L[f](s) = \int_0^{\infty} e^{-sx}f(x)dx $\\
    
    \small{
    	\begin{tabular}{lp{4cm} l}
        $f(t) = t^n, n \geq 0 $ &$F(s) = \frac{n!}{s^{n+1}}, s > 0 $ \\
        $f(t) = e^{at}, a \textit{ constant}$ & $ F(s) = \frac{1}{s-a}, s > a$ \\
        $f(t) = \sin{bt}, b \textit{ constant}$ & $ F(s) = \frac{b}{s^2 + b^2}, s > 0$ \\
        $f(t) = \cos{bt}, b \textit{ constant}$ & $ F(s) = \frac{s}{s^2 + b^2}, s > 0$ \\
        $f(t) = t^{-1/2}$ & $F(s) = \frac{\pi}{s^{1/2}}, s > 0$ \\
        $f(t) = \delta(t-a)$ & $F(s) = e^{-as}$ \\
        $f'$ & $L[f'] = sL[f] - f(0)$ \\
        $f''$ & $L[f''] = s^2 L[f] - sf(0) - f'(0)$ \\
        $L[e^{at}f(t)]$ & $L[f](s-a)$ \\
        $L[u_a(t)f(t-a)]$ & $L[f]e^{-as}$ 
        \end{tabular}}
    \end{minipage}
};
%------------ Laplace Transforms Header ---------------------
\node[fancytitle, right=10pt] at (box.north west) {3.1 Laplace Transform};
\end{tikzpicture}

%------------ Frequency-Response Method ---------------
\begin{tikzpicture}
\node [mybox] (box){%
    \begin{minipage}{0.3\textwidth}
    \small{
    	\begin{tabular}{lp{6cm} l}
    	\textit{Input} &   
    	$u(t) = A\cos{\omega t + \phi}$, \\ \hline
		\textit{Response} &    
		$y_{ss}(t) = A\vert G(j\omega)\vert cos(\omega t + \phi + \angle G(j \omega))$. \\ \hline
		\textit{Phase angle} &    
		$\angle G(j\omega) = \arctan{(\frac{\Im{G(j\omega)}}{\Re{G(j\omega)}})}$, \\ &
		$\angle G(j\omega) = \angle N\{G(j\omega)\} - \angle D\{G(j\omega)\}$. \\ \hline
	\end{tabular}}
    \textrm{For a LTI system without poles on the imaginary axis, i.e.,} $G_0(s) = 0$.
    \end{minipage}
};
%------------ Frequency-Response Method Header ---------------------
\node[fancytitle, right=10pt] at (box.north west) {3.8 Frequency Response};
\end{tikzpicture}

%------------ Tracking and Control ---------------
\begin{tikzpicture}
\node [mybox] (box){%
    \begin{minipage}{0.3\textwidth}
    \small {
    	\begin{tabular}{lp{4cm}}
    	\texit{Lemma 4.1} \qquad Asymptotic tracking \\ \hline
		Let $S(s)$ be the asymptotically stable transfer function \\ $S(0) = 0$. The system asymptotic tracks a constant refer-\\ence input if the product $GK$ has a pole at the origin. \\ \hline
    	\textit{Lemma 4.3} \qquad Internal stability \\ \hline
		The closed-loop system is internally stable if and only if \\ $S(s)$ is asymptotically stable and any pole-zero cancellat-\\ions of the product $GK$ satisfy $\Re\{z\} < 0$. \\ \hline
	\end{tabular}}
	\small{
	\begin{tabular}{lp{4cm}}
	    \textit{Asymptotic stability} &
		If $G(s)$ converges and is bounded for all $\Re(s) \geq 0$, i.e., $G(s)$ does not have poles on the imaginary axis or on the right-hand side of the complex plane. \\ \hline
	    \textit{Asymptotic tracking} &
		If $S(s)$ has a zero at $s = 0$ and the product $GK$ has a pole at the origin. \\ \hline
		\textit{Asymptotic} &
		If $D(s)$ has a zero at $s = 0$, &
		\textit{disturbance rejection} &
		or the controller $K(s)$ has a pole at the origin.\\ \hline
		\textit{Asymptotic tracking} &
		If $S(s)$ has a zero at $s = 0$.
	\end{tabular}}
    \end{minipage}
};
%------------ Tracking and Control ---------------------
\node[fancytitle, right=10pt] at (box.north west) {4.1 Tracking, Sensitivity and Integral Control};
\end{tikzpicture}

%------------ Feedback with Disturbances Content ---------------
\begin{tikzpicture}
\node [mybox] (box){%
    \begin{minipage}{0.3\textwidth}
    \small{
    	\begin{tabular}{lp{4cm} l}
		\textit{Sensitivity} & i.e., from $\bar{v_1} \textrm{   to} \textrm{   } e $ \\
		& $S = \frac{1}{1 + GK}$
		\\ \hline
		\textit{Disturbance} & i.e., from $w  \textrm{   to} \textrm{   } e$ \\
		& $ D = GS = \frac{G}{1 + GK}$
		\\ \hline
		\textit{Complementary sensitivity} & i.e., from $\bar{v_1} \textrm{   to} \textrm{   } v_1$ \\
		& $ H = GKS = \frac{GK}{1 + GK}. $
		\\ \hline
		\textit{Design} & $ Q = KS = \frac{K}{1 + GK}. $
		\\ \hline
		\\
		\textit{Closed-loop system} & \textit{For inputs} \\ \hline
		\\ $y = G(u + w)$,          & $y = H\bar{y} - Hv + Dw$, \\
        \\ $u = K\Tilde{e}$,        & $u = Q\bar{y} - Qv - Hw$. \\
        \\ $\Tilde{e} = \Tilde{y} - (y + v)$. & with measurement noise $v$.
		\\ \hline
		\textit{Tracking error} & $e = \bar{y} - y = S\bar{y} + Hv - Dw$.
	\end{tabular}}
    \end{minipage}
};
%------------ Feedback with Disturbances Header ---------------------
\node[fancytitle, right=10pt] at (box.north west) {4.4 Feedback with Disturbances};
\end{tikzpicture}

%------------ Pole-Zeros Content ---------------
\begin{tikzpicture}
\node [mybox] (box){%
    \begin{minipage}{0.3\textwidth}
    \small{
    	\begin{tabular}{lp{5cm} l} 
    	\textit{Sensitivity} &
        The poles of $S(s)$ are the zeros of the characteristic equation of $S(s)$. \\ &
        Furthermore, the zeros of $S(s)$ are the poles of the product $GK$.
    	\\ \hline
		\textit{GK} &
		$ \tilde{G}\tilde{K} = \frac{ N_{\tilde{G}} N_{\tilde{K}} }{ D_{\tilde{G}} D_{\tilde{K}} } $ which have polynomial roots $N_{\tilde{G}} N_{\tilde{K}} + D_{\tilde{G}} D_{\tilde{K}} = 0$. \\ \hline
		\textit{S} &
		If all the roots of $\tilde{G}\tilde{K}$ have negative real parts, then $S(s) = \frac{ D_{\tilde{G}} D_{\tilde{K}} }{ N_{\tilde{G}} N_{\tilde{K}} + D_{\tilde{G}} D_{\tilde{K}}}$ and $H = 1 - S$ are asymptotically stable. \\ \hline
		\textit{SG and SK} &
		$SG = \frac{(s - z)}{(s - p)} \frac{ N_{\tilde{G}} N_{\tilde{K}} }{ N_{\tilde{G}} N_{\tilde{K}} + D_{\tilde{G}} D_{\tilde{K}}}$, \\ &
		$SK = \frac{(s - p)}{(s - z)} \frac{ N_{\tilde{K}} D_{\tilde{G}} }{ N_{\tilde{G}} N_{\tilde{K}} + D_{\tilde{G}} D_{\tilde{K}}}$ \\&
		are stable only if $p$ and $z$ are both negative.
	\end{tabular}}
    \end{minipage}
};
%------------ Pole-Zeros Header ---------------------
\node[fancytitle, right=10pt] at (box.north west) {4.7 Pole-Zero Cancellations and Stability};
\end{tikzpicture}

%------------ State-Space Content ---------------
\begin{tikzpicture}
\node [mybox] (box){%
    \begin{minipage}{0.3\textwidth}
    \small{
    	\begin{tabular}{lp{5cm} l} 
    	\textit{State-space form} &
        $\dot{x} = Ax + Bu, y = Cx + Du.$
    	\\ \hline
		\textit{A} &
		The state matrix of size $n\times n,$ where $n$ is the number of state variables. \\ \hline
		\textit{B} &
		The input matrix of size $n\times m$, where $m$ is the number of outputs. \\ \hline
		\textit{C} &
		The output matrix of size $m\times n$. \\ \hline
		\textit{D} &
		The transition matrix of size $m\times m$. \\ \hline
		\textit{Equilibrium points} &
		Set partial derivatives of system equations to zero, i.e., $\frac{\partial y}{\partial x_1} = 0$.
	\end{tabular}}
    \end{minipage}
};
%------------ State-Space Header ---------------------
\node[fancytitle, right=10pt] at (box.north west) {5.2 State-Space Models};
\end{tikzpicture}

%------------ Second-Order Systems Content ---------------
\begin{tikzpicture}
\node [mybox] (box){%
    \begin{minipage}{0.3\textwidth}
    \small{
    	\begin{tabular}{lp{5cm} l} 
    	\textit{Characteristic} &
    	$s^2 + 2\zeta \omega_{n}s + \omega_{n}^2$, \quad $\omega_{n} > 0$. \\ \textit{equation} &
    	\\ \hline
    	\textit{Canonical} &
    	$(1 - \omega^{2}/\omega_{n}^{2} + j2\zeta\omega/\omega_{n})^{k}, \quad \vert \zeta \vert < 1$. \\ \textit{form} &
    	\\ \hline
    	\textit{Poles} &
    	$p = -\omega_{n}(\zeta \pm \sqrt{\zeta^2 - 1}), \Re\{p\} = -\omega_{n}, \quad \vert p \vert = \omega_{n} \left \vert -\zeta \pm j\sqrt{1-\zeta^2} \right\vert$ &
    	\\ \hline
		\textit{Inverse Laplace} &    
		$\mathscr{L}^{-1} = \left \{ \frac{k}{s + \zeta \omega_n - j\omega_d} + \frac{k^{*}}{s + \zeta \omega_n + j\omega_d} \right \} $ \\ \textit{Transform} &
		$ { } = 2\vert k \vert e^{-\zeta\omega_{n}t}\cos(w_{d}t + \angle k)$
		\\ \hline
		\textit{Parameters} &    
		$\omega_{d} = \omega_{n} \sqrt{1 - \zeta^2}$, \quad $0 < \zeta < 1$. \\ &
		$\phi_{d} = \arcsin\zeta = \arctan \left( \frac{\zeta \omega_{n}}{\omega_{d}} \right)$.
		\\ \hline
		\textit{Step response} &
		$G(s) = \frac{\omega_{n}^2}{s^2 + 2\zeta\omega_{n}s + \omega_{n}^2}$, \quad $\omega_{n} > 0$. \\ &
		$y(t) = \mathsrc{L}^{-1} \left\{ \frac{G(s)}{s}\right\}$, \\ &
		$\qquad = 1 - \frac{1}{\sqrt{1 - \zeta^2}}e^{-\zeta\omega_{n}t}\sin(\omega_{d}t + \pi/2 - \phi_{d})$
		\\ \hline
	\end{tabular}}
	where $\zeta$ is the \textit{damping ratio} and $\omega_n$ is the \textit{natural frequency}.
    \end{minipage}
};
%------------ Second-Order Systems Header ---------------------
\node[fancytitle, right=10pt] at (box.north west) {6.1 Second-Order Systems};
\end{tikzpicture}

%------------ Root-Locus Content ---------------
\begin{tikzpicture}
\node [mybox] (box){%
    \begin{minipage}{0.3\textwidth}
    \small{
    	\begin{tabular}{lp{4.5cm} l}
    	\textit{Characteristic} &
    	$1 + \alpha L(s)$, \quad $\alpha \geq 0$, \\ \textit{equation} & $L(s) = G(s)K(s)F(s).$
    	\\ \hline
    	\textit{Branches} &
    	Equal to the order of the characteristic equation, i.e., the denominator of the loop gain transfer function $L(s)$.
    	\\ \hline
    	\textit{Asymptotes} &
    	$N = n - m$, asymptotes/closed-loop poles for $n$ poles and $m$ zeros in the loop-gain transfer function. The asymptotes intersect the real axis at the point $c = \frac{\sum_{k=1}^{n}p_k - \sum_{i=1}^{m} z_i}{n-m}$ with angle $\angle \phi_{k} = \frac{\pi + 2\pi k}{n - m}$ for $k = 0,...,n-m-1.$ &
    	\\ \hline
    	\textit{Root-locus on} $\Re$ &
    	A point $s$ on the real axis will only be on the root-locus if it is to the left of an odd number of poles and zeros.
    	\\ \hline
		\textit{Break-away} &    
		$\textrm{CE} + k = 0, \quad k = -\textrm{CE}.$ \\ \textit{points} & $s$ is a break-away point if $\frac{dk}{ds} = 0$ is real and positive. 
		\\ \hline
		\textit{Angle of departure} &    
		Root-locus leaves a complex pole $p_j$ at $\theta_{d} = 180^{\circ} + \sum \angle (p_j - z_i) - {\sum \angle {p_j - p_i}}$.
		\\ \hline
		\textit{Angle of arrival} &    
		Root-locus arrives a complex zero $z_j$ at $\theta_{a} = 180^{\circ} - \sum \angle (z_j - z_i) - {\sum \angle {z_j - p_i}}$.
	\end{tabular}}
    \end{minipage}
};
%------------ Root-Locus Header ---------------------
\node[fancytitle, right=10pt] at (box.north west) {6.4 Root-Locus};
\end{tikzpicture}

%------------ Bode Plot Content ---------------
\begin{tikzpicture}
\node [mybox] (box){%
    \begin{minipage}{0.3\textwidth}
    $\textrm{Magnitude}$ \quad \qquad $20\log_{10}\left \vert G(j\omega) \right \vert,$ in units of $dB$.
    \small{
    	\begin{tabular}{lp{5.7cm} l} \hline
    	\textit{Gain offset}  &  $ 20\log_{10}\left \vert G(0) \right \vert$ for constant gain, $-20 dB$ for pole at origin, $+20 dB$ for zero at origin. \\ \hline
    	\textit{Gain margin} &  $GM = 0 - A_{m} dB$, The corresponding point on the magnitude curve at the phase crossover frequency, i.e., the vertical axis intersection point where $\angle G(j\omega) = \vert -180^{\circ} \vert$. \\ \hline
    	\textit{Poles} &  $(1 + \frac{s}{\tau})^{k} \rightarrow -20\times k \frac{dB}{dec}$ at $\omega = \tau$ if stable, else $+20\times k \frac{dB}{dec}$. If second-order, $(\frac{1}{(1+2\zeta/\omega_n)}j\omega + \frac{1}{\omega_n^2}(j\omega)^2) \rightarrow -40 \frac{dB}{dec}$ at $\omega = \omega_n$. \\ \hline
    	\textit{Zeros} & $(1 + \frac{s}{\tau})^{k} \rightarrow +20\times k \frac{dB}{dec}$ at $\omega = \tau$ if stable, else $-20\times k \frac{dB}{dec}$. If second-order, i.e., $(\frac{1}{(1+2\zeta/\omega_n)}j\omega + \frac{1}{\omega_n^2}(j\omega)^2) \rightarrow +40 \frac{dB}{dec}$ at $\omega = \omega_n$. \\ \hline
    	\textit{Slope} & Ends with $-20\times (n - m) \frac{dB}{dec}$. \\ \hline
    	\textit{Unit gain} & $\beta$, such that $\beta \vert G(0) \vert = 1$, i.e., substitute $s = j\omega = 0$ for poles/zeros of $G(s)$ not at the origin, and $\beta\vert G(j1) = 1$, i.e., $s = j1$ for poles/zeros at the origin, then solve for the value of $\beta$ that makes the expression of magnitude $\vert G(j\omega) \vert$ true. \\ \hline
    \end{tabular}}
    \newline
    \newline
    $\textrm{Phase}$ \quad \qquad \qquad $\angle G(j\omega),$ in units of $degrees$.
    \small{
    	\begin{tabular}{lp{5.7cm} l} \hline
    	\textit{Gain offset} &  $+180^{\circ}$ if negative, $+90^{\circ}$ for each pole, $-90^{\circ}$ zero at origin. \\ \hline
    	\textit{Phase margin} &  $PM = \phi_{m} - (-180^{\circ})$, the corresponding point on the phase curve at the gain crossover frequency, i.e., the vertical axis intersection point where $\vert G(j\omega) \vert = 0 dB$. \\ \hline
    	\textit{Poles} &        $(1 + \frac{s}{\tau})^{k} \rightarrow -45^{\circ}\times k$ at $\omega = 0.1\tau$ if stable, else $+45^{\circ}\times k$. If second-order, i.e., $\left(\frac{1}{(1+2\zeta/\omega_n)}j\omega + \frac{1}{\omega_n^2}(j\omega)^2\right) \rightarrow -\tan^{-1}\left(\frac{\frac{2\zeta\omega}{\omega_{n}}}{1-\left(\frac{\omega}{\omega_{n}}\right)^2}\right) \frac{dB}{dec}$ at $\omega = 0.1\omega_n$. 
    	\\ \hline
    	\textit{Zeros} &        $(1 + \frac{s}{\tau})^{k} = +45^{\circ}\times k$ at $\omega = 0.1\tau$ if stable, else $-45^{\circ}\times k$. If second-order, i.e., $\left(\frac{1}{(1+2\zeta/\omega_n)}j\omega + \frac{1}{\omega_n^2}(j\omega)^2\right) \rightarrow +\tan^{-1}\left(\frac{\frac{2\zeta\omega}{\omega_{n}}}{1-\left(\frac{\omega}{\omega_{n}}\right)^2}\right) \frac{dB}{dec}$ at $\omega = 0.1\omega_n$. \\ \hline
    	\textit{Slope} &        ends flat at $\angle 90^{\circ}\times(m_{-} + m_{0} + n_{+}) - 90^{\circ}\times(m_{+} + n_{-} + n_{0})$, \\ \hline
    \end{tabular}}
    where $m_{+}/n_{+}$ are number of unstable poles/zeros, $m_{-}/ n_{-}$ stable poles/zeros, $m_{0}/n_{0}$ poles/zeros on imaginary axis.
    \end{minipage}
};
%------------ Bode Plot Header ---------------------
\node[fancytitle, right=10pt] at (box.north west) {7.1 Bode Plot};
\end{tikzpicture}

%------------ Nyquist Stability Content ---------------
\begin{tikzpicture}
\node [mybox] (box){%
    \begin{minipage}{0.3\textwidth}
        The curve of $G(j\omega)$ parameterized by the value $\omega \in \Re$ describes the polar plot of the system, i.e., the phase and magnitude on the real and complex plane. The polar plot has a contour $\Gamma$ covering the entire right-hand side of the complex plane from the limit $\rho \rightarrow \infty$.
    \small{
    	\begin{tabular}{lp{5.5cm} l} \hline
    	\textit{Theorem 7.1}  &  If a function $f$ is analytical inside and on the the positively oriented simple closed contour $C$ except at a finite number of poles inside $C$ and $f$ has no zeros on $C$ then
    	$\frac{1}{2\pi}\Delta_{C}^{0}\textrm{arg}f(s) = \frac{1}{2\pi j} \int_{C}^{} \frac{f'(s)}{f(s)}ds = Z_{C} - P_{c}$,
    	where $Z_{C}$ is the number of zeros, $P_{C}$ poles inside contour $C$ counting their multiplicities. \\ \hline
    	\textit{Theorem 7.2}  &  Assume that a transfer function $L(s)$ has no poles on the imaginary axis. For any given $\alpha > 0$ the loop transfer function in negative feedback with static gain controller $\alpha$ is asymptotically stable if and only if the number of counter-clockwise encirclements of the image contour $\Gamma$ around the point $-1/\alpha$ is equal to the number of poles of $L(s)$ on the right-hand side of the complex plane.\\ \hline
    	\textit{Closed-loop}   & The closed-loop system $L(s)$ is \\ \textit{stability} & asymptotically stable if and only if $Z_{\Gamma} = P_{\Gamma} - \frac{1}{2\pi}\Delta_{\Gamma}^{-\frac{1}{\alpha}} \textrm{arg}L(s) = 0$. \\ \hline
    \end{tabular}}
    where $Z_{\Gamma}$ is the number of closed-loop poles and $P_{\Gamma}$ the number of open-loop poles of $L(s)$ inside the contour $\Gamma$, i.e., on the RHP.
    \end{minipage}
};
%------------ Nyquist Header ---------------------
\node[fancytitle, right=10pt] at (box.north west) {7.6 Nyquist Stability Criterion};
\end{tikzpicture}

%------------ Complex Numbers Content ---------------
\begin{tikzpicture}
\node [mybox] (box){%
    \begin{minipage}{0.3\textwidth}
        For a complex number in the form $a + jb$,
    \small{
    	\begin{tabular}{lp{5.5cm} l} \hline
    	\textit{Amplitude}  &  $\vert A \vert = \sqrt{a^2 - (jb)^2} = \sqrt{a^2 + b^2}$, \\ & where $j^2 = -1$. \\ \hline
    	\textit{Phase} &  $\phi = \arctan(\left \frac{b}{a} \right)$, where $\arctan(0) = 0^{\circ}, \arctan(1) = 45^{\circ}, \arctan(\infty) = 90^{\circ}$.
    \end{tabular}}
    \end{minipage}
};
%------------ Complex Numbers Header ---------------------
\node[fancytitle, right=10pt] at (box.north west) {Complex Numbers};
\end{tikzpicture}
\end{multicols*}
\end{document}